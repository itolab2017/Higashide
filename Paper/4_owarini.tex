\chapter{おわりに}
既存のITOLAB MOTORDRIVERを基に改良することで,短期間でモータドライバを製作すること
ができた.
また,KiCadを用いることで,設計時間の短縮ができた.

今後,電流センサ,温度センサの検出値の確認,RXマイコンによるシリアル通信の確認が必要
である.
